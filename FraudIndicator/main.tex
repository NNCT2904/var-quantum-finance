\section{Anomaly Detection and Fraud Indicator} \label{Sec: Anomaly Detection and Fraud Indicator}

Anomaly detection is used to find the data that is not follow the pattern of the given dataset.
Usage of fraud detection is diversed in many important application such as Fraud Detection, Data cleaning and surveillance.
The unusual data pattern, or anomalies, can be caused of unauthorised intrusion into a system, or faulty system.

Machine learning algorithms are indeed play an pivot roles in anomaly detection for classical data in form of quantum state using quantum devices.
These algorithms typically provide measurement of how far the inspepcted data astray from the normal pattern.
Widely used methods can include: Quantum Support Vector Machine (QSVM) \cite{grossiMixedQuantumClassical2022} and Principal Component Analysis (PCA) \cite{lloydQuantumPrincipalComponent2014, liuQuantumMachineLearning2018}.

\subsection{Quantum Support Vector Machine}
Michele et al. \cite{grossiMixedQuantumClassical2022} had presented the first end-to-end application of QSVM for financial industry based on real payment data.
The author explored the approach to utilise QSVM for better improve the fraud detection.
