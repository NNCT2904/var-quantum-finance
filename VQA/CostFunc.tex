\subsection{The Cost Function} \label{Sec: The Cost Function}\

% Encoding the problem into a cost function is the first step in solving a problem using VQA \cite{cerezo2021variational}.
% The cost function is equivalent to that used in classical machine learning.
% It maps the values of the trainable parameters $\theta$ into real values, which represent the measure of distance from an optimum solution.
% For a function $f$ that receives input states $\{\rho_k\}$, observables $\{O_k\}$, and a parameterized circuit $U(\theta)$, the cost is expressed as:

Both classical machine learning and quantum machine learning use the cost function to map the the trainable parameters $\theta$ into a value that represent the distance from the optimal solution.
This is the first step in finding a solution using VQA \cite{cerezo_variational_2021}.
For a function $f$ with the input states $\{\rho_k\}$, the observables $\{O_k\}$ and the parameterised circuit $U(\theta)$, the cost is as following equation:

\begin{equation}
    C(\theta) = f(\{\rho_k\}, \{O_k\}, U(\theta)) \;,
\end{equation}
or this equation with a set of functions $\{f_k\}$ and the trace $Tr$ as the square of a distance matrix:
\begin{equation}
    C(\theta) = \sum_k f_k \left(\Tr[ O_k U(\theta) \rho_k U^\dagger(\theta) ]\right) \;,
    \label{Eqn: Cost function}
\end{equation}

A cost function of an VQA algorithm must meet a number of criteria:
(1) The cost function must be 'faithful' and 'operationally meaningful', the minimum value of $C(\theta)$ should correspond to the optimal solution of the problem, and we can expect the better solution with lower cost function value in general;
(2) The cost function must be 'efficiently estimated' on a quantum computer by measuring qubits values;
(3) The cost function must be 'trainable', such that the parameters $\theta$ could be efficiently optimised.