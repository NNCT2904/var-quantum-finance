\section{Introduction} \label{Sec: Introduction}

\emph{Quantum computing} is the area of research that involves physics, computer science and mathematices.
Theoretically, the computational power of a quantum processor would scale exponentially with the number of qubits, and eventually surpasses the classical computers.
For this reason, many algorithms for quantum computers have been developed to solve problems that are challenging for classical computers, eg. the problems that belong to the non-deterministic polynomial time class \cite{williamsSolvingNPCompleteProblems2011,jiangQuantumAnnealingPrime2018,farhiQuantumApproximateOptimization2014}.
Such algorithms are understood as the unitary matrices to perform operations on qubits' values, and are often represented as \emph{quantum gates} on a \emph{quantum circuit}.

In practical, available quantum hardware are severely constrainted in resourse due to decoherence, and their applications are thus limited.
They are all described as Noisy Intermediate-Scale Quantum (or NISQ) computers \cite{brooksQuantumSupremacyHunt2019}.
In other words, we can expect the technical maturity of nowadays quantum devices are comparable to the first computers one hundred years ago.
At the current stage, all NISQ devices face the issue of precision for gate operations, the lack of fault-tolerance design to handle data integrity in case of decoherence, the limitation in the number of qubits in a processor, and the number of gates allowed for those qubits (the size of the circuit).

An approach such as Variational Quantum Algorithms (VQA) \cite{cerezo_variational_2021} is used to address this issue, by adopting the machine learning models to produce quantum circuits.
This technique involves the quantum circuits as the machine learning models that can receive trainable parameters, which act as a reusable template.
The classical component of the VQA model can feed and train the trainable parameters with any optimisation algorithm available.

VQA is also the mainstream way to design Quantum Neural Network (QNN) \cite{altaisky2001quantum}since they all share the same structure of classical optimisers and quantum ciurcits.
Typical QNN will have the three traits as pointed out by Schuld et al \cite{schuldQuestQuantumNeural2014}: (1) Initial state can encode the data as binary string; (2) The caculation process is based on Neural Network; (3) The system evolution is based on quantum theory.


There are many problems in quantitative finance belongs to the NP-hard class for classical computer.
Such problems required extensive computational power and thus may be benefit from quantum processors.
In this paper, we review the state of the art of quantum machine learning solutions for application in financial sector, especially Portfolio Optimisation, Price/Option Prediction, Fraud Indicators, Credit Scoring, Financial Risk Assessment.

% In our discussion we will assume that the readers have some background knowledge of quantum computing, linear algebra, and machine learning.
% To gain such a background knowledge, we recommend the 2020 Qiskit Global Summer School course \cite{2020QiskitGlobal}, which provides some basic mathematics, theories and practices of quantum computing.
% The book by Sutor \cite{sutorDancingQubitsHow2019} also provides the foundations of quantum Computing plus an overview of issues related to the design of current quantum hardware and simulators.
% Finally, the 2021 Qiskit Global Summer School course \cite{2021QiskitGlobal} was focused on the selected aspects of machine learning for quantum devices.